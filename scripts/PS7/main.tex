\documentclass{article}
\usepackage{graphicx} 
\usepackage{amsmath}


\title{Modeling the Living Cell HW 7}
\author{Joshua Khorsandi}

\begin{document}

\maketitle

\section*{Problem 1}
\begin{enumerate}
    \item There are 9 species to keep track of 
    \begin{enumerate}
        \item geneA
        \item mRNA A 
        \item geneA bound
        \item A 
        \item geneR
        \item geneR bound
        \item mRNA R 
        \item R 
        \item C
    \end{enumerate}
    \item There are 16 reactions in total, 14 reactions plus the 2 reversible reactions.
    \item Reactions 10, 11, 12, 13, and 14 involve the degredation of a species.
    \item 9, 2a (the forward reaction) and 5a both involve bimolecular association. Of these the formation of the C complex occurs the fastest.
    \item geneA exists both on its own and in its bound complex, the bound complex forms when A binds to geneA, so the formation of A (which occurs only from the bound complex), is a positive feedback loop, where excess A promotes its own formation.
    \item geneR is transcribed to mRNA more frequently. 
    \item mRNA R is translated into protein more frequently.
    \item R degrades the most frequently.
\end{enumerate}

\end{document}
